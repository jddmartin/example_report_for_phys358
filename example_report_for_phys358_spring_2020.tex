% an example latex source file
\documentclass[11pt]{article}
\usepackage[left=1cm, top=1cm, right=3cm, bottom=1cm, marginparwidth=2cm]{geometry}
\pagestyle{empty}
\begin{document}

\begin{center}
  \large\textbf{PHYS 342, Spring 2020: Report 3} \\ \vspace{10 pt}
  \textbf{Robie Hennigar (with contributions from Ali Ramadhan)} \\
  \normalsize ID: 13333337 \\
\end{center}

I came away from this assignment with much more confidence in my understanding of Ampere's law and improved my ability to deduce the ``shape" of a magnetic field produced by a given current distribution.  However, as I tackled some of the later problems, I realized that my multivariate calculus is weaker than I had anticipated.  I realize this a crucial component of the course, so I have made plans with a classmate to review the relevant sections in the first chapter of Griffiths this weekend. I think it might benefit future students if the assignment had some warm-up questions from the first chapter to remind us of the most important concepts from multivariate calculus. Below I will go into more detail about the specifics of the assignment.

I was able to complete problems (1) through (3) without referring to the solutions for help. In these problems I benefited from having a good working understanding of Ampere's law. I found problem (3) part (c) particularly interesting, since the result is the speed of light, which I had not anticipated.  I was wondering, \textbf{ is there any reason why I should expect such a result on physical grounds?}

I struggled with part (d) of problem (4) because it was unclear to me how to obtain a current density here.  I was able to produce the correct answer after referring to the solutions, but I did not understand, for example, the equality made in the second line on page (3) of this solution: that $I/v = \lambda$.  A friend told me that it follows from dimensional analysis, which I now understand, but I was wondering: \textbf{ is there is any way to understand this result without resorting to dimensional analysis?}

I made a silly mistake when first attempting problem (5) -- I forgot to use $\nabla$ in cylindrical coordinates. Otherwise it was an easy solve. I understand that the forms for the derivative operator in different coordinates will be provided on the midterm and final exam, otherwise I would definitely be putting this on my cheat sheet!

Problem (6) provided useful experience with vector calculus identities.  However, it also made me realize that I am just not comfortable with these yet, especially when they involve the derivative operator.  After investing some time---and making a few mistakes---I was able to obtain the correct answer.  I did not understand the second part of the problem about the uniqueness of the vector potential, though.  I can understand the solutions, but I have to admit I would have never thought to add the particular choice of function to $\vec{A}$ that is done in the last few lines of the solution. Maybe the solutions can provide some motivation for this step?

I completed problem (7) without difficulty.  In fact, it made me feel clever since I realized that the result from the previous question could be applied here.

I had a good deal of trouble with problem (8) and had to follow the solutions directly in order to make progress.  Even then there were aspects I did not understand---for example, why we could set terms such as $\vec{A}(0,y,z)=0$ (see eq. (12) in the solution to this question).  I am unsure if I truly understand this problem, or was just able to follow the solutions.  I am going to wait a few days and then tackle this problem again to see if I can now complete it without referring to the solutions along the way. I feel like the solutions could be made clearer using the TA's explanation.

I did not have time to complete part (c) of problem 8 nor any of problem (9).  I will be working on these over the weekend.

For the next assignment, I am going to work on one problem a day (using a fixed amount of time), to improve my long-term retention of the material, rather than waiting until closer to the deadline.

I am still unsure if the magnetic vector potential has a physical interpretation or if it's just a useful mathematical construction. I would appreciate it if this was covered in the notes more clearly or if there was an assignment question addressing this.
\end{document}
